\documentclass[journal, a4paper]{IEEEtran}

\usepackage{graphicx}   
%\usepackage{subfigure}
\usepackage{url}        
\usepackage{amsmath}    
% Some useful/example abbreviations for writing math
\newcommand{\argmax}{\operatornamewithlimits{argmax}}
\newcommand{\argmin}{\operatornamewithlimits{argmin}}
\newcommand{\x}{\mathbf{x}}
\newcommand{\y}{\mathbf{y}}
\newcommand{\ypred}{\mathbf{\hat y}}

\begin{document}

% Define document title, do NOT write author names
\title{The Title of your Report}
\author{Anonymous Authors}
\maketitle

% Write abstract here
\begin{abstract}
	A short summary of your project. You should change also the title, but do \emph{not} enter any author names or anything that unnecessarily identifies any of the authors. It is suggested you use a similar structure (sections, etc.) as demonstrated in this document, but you can make the section headings more descriptive if you wish. Of course \emph{you should delete all the text in this template and write your own}! -- this text simply provides detailed instructions/hints on how to proceed.

\end{abstract}

% Each section begins with a \section{title} command
\section{Introduction}

Describe what you did. Provide access to your anonymized code\footnote{Our code is available here: \url{http://anonymouslinktoyourcode.zip}}.

Note that results should be reproducible using the technologies from the labs (i.e., Python, and selecting among Scikit-Learn, OpenAI Gym, TensorFlow, PyGame, \ldots).

Do not change the formatting (columns, margins, etc). Hint: shared tools like \texttt{http://sharelatex.com/} and \texttt{http://overleaf.com/} are great tools for collaborating on a multi-author report in latex. If you wish to use Word, base it on the IEEE template\footnote{\url{https://www.ieee.org/publications_standards/publications/conferences/2014_04_msw_a4_format.doc}} and convert to \texttt{pdf} for submission. 

\section{Background and Related Work}

Elaborate (in your own words) the background material required to understand your work. It should cover a subset of the topics touched upon in the course. You are encouraged to cite topics in lectures, e.g., structured output prediction in \cite{LectureSOP}, book chapters, e.g., Chapter 9 from \cite{Barber}, or articles from the literature, e.g., \cite{Astar,DeepMindSC2}. Basically, you should prepare the reader to understand what you are about to present in the following sections. Eq.~\eqref{eq:MAP} shows a random equation.
\begin{equation}
	\label{eq:MAP}
	% Note the example \newcommand s defined above which make it faster to write latex math
	\ypred = \argmax_{\y \in \{0,1\}} p(\y|\x)
\end{equation}

\section{The Environment}

Describe your environment, either one you adapted/borrowed from somewhere, or designed yourself. Convince the reader that it is an interesting and/or challenging environment (could it potentially have real-world use or is based on real-world data? Or simply to provide an interesting/fun/challenging problem to tackle. In particular you should outline the particular challenges it poses as a RL problem.

\section{The Agent}

The agent you designed for your environment. Justify your choice and design and explain briefly how you implemented/configured it. Naturally, if you took a ready-made environment, you should invert relatively much more effort into this section than the previous one.

\section{Results and Discussion}

To measure how well the two agents perform in our environment, we use several measurements.
We introduce the \emph{Random Agent}, which chooses at every step a random direction.
We simulate the matches on a grid of size $30$, with $10$ candies on the map and the two adversarial agents we want to compare.

\subsection{Performance of your Agent in your Environment}

\subsubsection{Performance of Learning}

\begin{figure}[h]
	\centering
    \includegraphics[width=0.8\columnwidth]{images/learning_curve_against_random.pdf}
    \caption{\label{learning_curve_against_random}The learning curve of the RL agent against a random one.}
\end{figure}

\subsubsection{Performance of the agents}

\begin{table}[h]
	\caption{\label{comparative_table}Results of the matches.}
	\centering
	\begin{tabular}{llll}
		\hline
        \textbf{Match} & \textbf{Minimax} & \textbf{Random} \\
		\hline
        \textbf{RL} & 0 - 0 & 4744 - 113 \\
        \textbf{Minimax} & 0 - 0 & - \\
		\hline
	\end{tabular}
\end{table}

\subsection{Performance of your Agent in the ALife Environment}

You deploy your agent in the ALife\footnote{\url{https://github.com/jmread/alife}} environment (a random screenshot shown in Figure~\ref{a_figure}). Does it work well? Why? Why not? Justify the adaptation you think is best.


\section{Conclusion and Future Work}
	This section summarizes the paper: Your environment and agent, its strength and its weaknesses. Also remark about what would be the next steps you would take if you or someone else were to continue/extend this project. 
	Note that for the initial submission you are limited strictly to 4 pages (double column), \emph{not including references}. An extra page will be allowed for final submission (after the initial reviews). 

% The bibliography:
\begin{thebibliography}{4}

	\bibitem{Barber} % Book
	D.~Barber. Bayesian Reasoning and Machine Learning,
	{\em Cambridge University Press}, 2012.

	\bibitem{LectureSOP} % Web document
		J.~Read. Lecture III - Structured Output Prediction and Search. \textit{INF581 Advanced Topics in Artificial Intelligence}, 2018.

	\bibitem{Astar}
	D.~Mena et al. A family of admissible heuristics for A* to perform inference in probabilistic classifier chains.
	{\em Machine Learning}, vol. 106, no. 1, pp 143-169, 2017.

	\bibitem{DeepMindSC2}
	O.~Vinyals et al. StarCraft {II:} {A} New Challenge for Reinforcement Learning.
	\url{https://arxiv.org/abs/1708.04782}, 2017. 

\end{thebibliography}

% Your document ends here!
\end{document}
